Tweets are a common way for political candidates to pronounce themselves about some topic. However, the use of jargon, open ended language and lack of context turn most of the direct classification methods useless. However, cues of the political and twitter realm can be used to extract features that can improve sentiment analysis substantially. In this work we will follow mainly the approach proposed by \parencite{2002a} of using machine learning classifiers but also will in some way give a limited amount of importance to the gramatical structures. We first extract common sense features, that hereby are called \emph{special tokens}. Then, we merge those features with bigrams of special tokens that probably take precedence in each of the attitudes. Finally, word2vec attributes are used to compute sentence orientation regardless of word order. Word2vec features can preserve syntactical and semantical relationships between words. The main contribution of this work is to be a proof-of-concept for extracting syntactic grammatical and semantic features on the domain of actual Mexican policy that improve automatic tweet attitude tagging. A simple special tokens bigram counts and wrod2vec features have shown an improvement over bag-of-word of y on average and an improvement of 0.07 and 0.20 respectively.